%----------------------------------------------------------------------------------------
% WORK EXPERIENCE SECTION
%----------------------------------------------------------------------------------------
\begin{rSection}{Professional experience}

  \begin{rSubsection}{Orange Application for Business}{\em June 2017 -- August 2017}{System administrator, Securing Workstations}{Bordeaux, France}
    \item[]
      OAB is a subsidiary of Orange S.A developing and hosting services for businesses.
    \item[]
      Integrated to an operations team of four, my mission was to secure their Linux workstations.

    \item System administration: Arch, Ubuntu, Debian.
    \item Define and apply good security practices for laptops and mobile devices.
    \item Development: ansible roles, \href{https://github.com/multimediabs/audit-ssh-client-config}{audit-ssh-client-config} script.
    \item Research: securing the boot process (anti-evil maid), Nixos, Public Key Infrastructures.
    \item Training and communication about the security practices: write documentation, create tutorials, run workshops.
  \end{rSubsection}

  %------------------------------------------------

  \begin{rSubsection}{RedShield Security Ltd}{\em January 2015 -- April 2016}{Software engineer}{Wellington, New Zealand}
    \item[]
      Shielding-with-a-service company providing a web application firewall, vulnerability scanner and analyzer to assist IT departments ensure the security of their systems.

    \item Development in Ruby and Javascript.
    \item Configuration of production and CI servers (Jenkins).
    \item Debugging, automated testing.
    \item Research and development.
    \item Security: offensive, secure development, incident response.
  \end{rSubsection}

  %------------------------------------------------

  \begin{rSubsection}{CER SNCF}{\em June 2012 -- Oct 2014}{Full-stack developer}{Bordeaux, France}
    \item[] Distributed all over the Aquitaine and Poitou Charentes French regions, the CER (for Committee of Regionnal Establishment) provides services to SNCF's workers.

    \textbf{First missions}
      \item Web development in PHP with the Zend framework.
      \item Production environment management: Nginx front-end with FCGI, PostgreSQL production database.
      \item Team work with Git and automated deployment on top of this VCS tool.
      \item Adaptation of our tool for another CER: needs analysis, objectives definition, integration to our development and deployment processes and tooling.

    \textbf{Project, subject of my master thesis}
      \item Build a website dedicated to SNCF's employees: tasks and planning definition, resources and skills affectation to provide this new service within six month with a three person team.
      \item The application can be found at \href{http://cheminot.cer-sncf-bx.org}{cheminot.cer-sncf-bx.org} .
      \item Communication and product launching: define and reach targets, follow the adoption and usage of the product.
      \item Integration within the existing information system: shared databases services, web server deployment.
      \item Quality management by software testing using Rspec and Capybara.
      \item Git and Capistrano for code management and automated deployment.
      \item Back-end written in Ruby-On-Rails. Front-end used Slim, SASS and CoffeeScript.
      \item Server on Ubuntu 14.04, database is PostgreSQL, Nginx and Unicorn as web stack.

    \textbf{Others}
      \item Information system maintenance: Zimbra mail, Windows2008 domain, Debian web servers and support.
      \item IT service management: needs analysis, tasks definition, listing of skills and resources to prioritize our work.
      \item Help in recruitment for IT service: define required skills, job scope ; participate during interviews.

  \end{rSubsection}

  %------------------------------------------------

  \begin{rSubsection}{Cabinet P.PI/4}{\em Summer 2011}{Web developer intern}{Bordeaux, France}
    \item Web development in PHP with the CodeIgniter framework.
    \item Seven students team organized with the Scrum method.
    \item Interviews with client for needs analysis.
  \end{rSubsection}

\end{rSection}
